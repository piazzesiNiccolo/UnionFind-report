\documentclass{article}
\usepackage[utf8]{inputenc}
\usepackage[T1]{fontenc}
\usepackage{babel}
\usepackage{listings}
\usepackage{xcolor}
\usepackage{tikz}
\lstset{language=C++,
                basicstyle=\ttfamily,
                keywordstyle=\color{blue}\ttfamily,
                stringstyle=\color{red}\ttfamily,
                commentstyle=\color{green}\ttfamily,
                morecomment=[l][\color{magenta}]{\#}
}

\begin{document}
\title{ \textbf{Union Find}}
\author{Niccolò Piazzesi\\n.piazzesi@studenti.unipi.it}
\date{}
\maketitle
\begin{abstract}
The Union Find data structure, also called Disjoint Set Union (DSU) is a very interesting data structure, not 
only  for the algorithmic techniques and  analysis involved, but also for its applications, especially to graph problems. In this
report I will present a high level view of UnionFind, describing the fundamental structure and the various implementations. In the last section, I will also present
a collection of problems for which the use of a DSU provides an efficient solution.
\end{abstract}
\section{Description}
Suppose we have a collection S of \emph{n} distinct elements, let's say the integers from 1 to \emph{n}. 

Initially these elements
are organized in n disjointed sets \{1\},\{2\},...,\{n\}. For simplicity we will assume that the name of set \{i\}
is i. In the union find problem we want to mantain a collection of disjointed sets while supporting the following operations:
\begin{itemize}
    \item \textbf{union}(A, B): join the sets A and B in a single set. The name of the new set is A and the old sets A and B are destroyed.
    \item \textbf{find}($x$): given an element $x$, return the name of the set that contains it.
\end{itemize}
Figure ** shows an example  with n=6. Let's generalize the problem by removing the assumption that we have n elements a priori.
Initially there are no elements and no sets, and the the new elements are introduced by the operation:\begin{itemize}
    \item makeSet($x$): Create a new set {x}. The name of the new set is $x$.
\end{itemize}
We will say that the UnionFind problem consists of mantaining a collection of disjointed sets during 
an arbitrary sequence of \emph{makeSet, union} and \emph{find} operations, starting from the empty set \cite{demetrescu}.
\section{Naive Implementations}
\subsection{QuickFind}

\subsection{QuickUnion}
\lstinputlisting[language=C++,caption=test,label=test]{code/test.cpp} 
\section{Balancing Heuristics}
\section{Time complexity and proof}
\section{Applications}
\bibliographystyle{plain}
\bibliography{bib}
\end{document}